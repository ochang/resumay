\documentclass[12pt,letterpaper,sans]{moderncv}
\moderncvstyle{classic}
\moderncvcolor{blue}
\usepackage[scale=0.88]{geometry}

\firstname{Oliver}
\familyname{Chang}

\title{Curriculum Vit\ae}
\mobile{727-771-3641}
\email{oliver@oychang.com}

% Format of cventry:
% \cventry{years}{degree/job title}{institution/employer}{localization}{grade}{description}

\usepackage{url}

\begin{document}


\makecvtitle


\section{Education}

\cventry{2012--Present}{Bachelor of Science, Computer Science \& Geography}{University of Miami}{Coral Gables, FL}{\textit{3.8 GPA}}{Mathematics Minor and Certificate in Geographic Information Systems}


\section{Research Experience}

\cventry{2015--Present}{Research Experience for Undergraduates}{A WiFi Localization-based Approach to Analyzing Probe Requests}{Florida International University}{Miami, FL}{%
\begin{itemize}
\item Captured WiFi probe requests with a WiFi Pineapple to use with ad hoc localization strategies to passively place users in environments
\item Built upon an openly developed application to create an extensible platform for the research of WiFi localization techniques
\end{itemize}}

\cventry{Summer 2015}{Research Experience for Undergraduates}{Optimization Opportunities for nek5000 on Blue Waters}{Parallel Computing Institute, University of Illinois, Urbana-Champaign}{Urbana, IL}{%
\begin{itemize}
\item Profiled nek5000, a Fortran MPI fluid dynamics simulator, to evaluate calculation/communication optimizations
\item Pushed scaling limits of nek software against Blue Waters' hardware to achieve a 5\% speedup on PGI compilers
\end{itemize}}

\cventry{Spring 2015}{Research Experience for Undergraduates}{Localization through WiFi Signal Strength with Android}{Florida International University}{Miami, FL}{%
\begin{itemize}
\item Worked collaboratively with two other students to create localization tools based on WiFi and LTE alternatives to GPS
\item Created Android app for training \& localization via WiFi using modern best practices like Fragment \& DataProvider
\end{itemize}}

\cventry{Summer 2014}{Summer Internship Program}{Automating Metadata Compliance Checking}{Physical Oceanography Distributed Active Archive Center, NASA Jet Propulsion Laboratory}{Pasadena, CA}{%
\begin{itemize}
\item Consolidated existing metadata checkers for netCDF into a Python framework with verifiable suggestions
\item Lowered cost \& difficulty of compliance by creating a RESTful platform over a dependency-heavy local installation
\end{itemize}}


\section{Professional Experience}

\cventry{2014--2015}{Undergraduate Teaching Assistant}{Computer Science Department, University of Miami}{Coral Gables, FL}{}{%
 \begin{itemize}
  \item Reviewed Java code for Intro to Programming with an emphasis on idiomatic, readable code
  \item Challenged small labs of Intro to Data Structures \& Algorithms students with visualizations emphasizing edge cases`
 \end{itemize}}

\cventry{2013--2014}{Software Engineering Intern}{Senzari, Inc.}{Miami, FL}{}{%
 \begin{itemize}
  \item Led visualization for a Django dashboard written with Google Charts \& CoffeeScript used to track visitor analytics
  \item Built one of the first Firefox OS Apps in HTML5/jQuery Mobile to construct graph queries, once rated 4 stars
 \end{itemize}}


\section{Publications and Presentations}

\cventry{2015}{\href{https://oychang.com/research/gisday/gps-wifi-poster-2015.pdf}{GPS \& WiFi Choke Point Analysis}}{Poster}{University of Miami GIS Day}{}{}
\cventry{2015}{\href{https://oychang.com/research/uiuc/nek-poster-final.pdf}{Analyzing the Scalability of Nek5000}}{Poster}{Summer Research Poster Session}{}{Presented in open house to University of Illinois, Urbana-Champaign community}
\cventry{2014}{\href{https://oychang.com/research/jpl/compliance-checker-poster.pdf}{Improving Compliance for Earth Science Data Records}}{Poster}{Fall Meeting of American Geophysical Union}{}{}
\cventry{2014}{\href{https://oychang.com/research/jpl/compliance-checker-presentation.pdf}{Automating Metadata Compliance Checking}}{Presentation}{Research Presentation Day}{}{Final Presentation for Jet Propulsion Laboratory Summer Internship Program}
\cventry{2014}{\href{https://oychang.com/research/presentation/index.html}{Adventures in Cheminformatics}}{Presentation}{University of Miami Computer Science Department}{}{Final Presentation for fulfillment of Computer Science Project Planning course}


\section{Technical Skills}

\cvitem{Basic}{Fortran, MPI, Ruby, R, UML Modeling, API Design}
\cvitem{Intermediate}{JavaScript, C, \LaTeX, Unit Testing, SQL}
\cvitem{Advanced}{Python, Java, Unix, Android Software Development Kit}


\section{Honors}

\cvitem{Nov 2015}{2nd place GIS Day undergraduate poster}
\cvitem{Fall 2015}{Supported by REU Supplement to NSF Grant \href{http://www.nsf.gov/awardsearch/showAward?AWD\_ID=1446570}{CNS-1446570}}
\cvitem{Summer 2015}{Supported by NSF grant \href{http://www.nsf.gov/awardsearch/showAward?AWD\_ID=1263145}{CCF-1263145}}
\cvitem{Spring 2015}{Supported by REU Supplement to NSF Grant \href{https://nsf.gov/awardsearch/showAward?AWD\_ID=1406968}{CNS-1406968}}
\cvitem{2012}{National Merit Scholarship}


\section{Service}

\cvitem{2015--Present}{Vice President for University of Miami (UM) chapter of Association for Computing Machinery}
\cvitem{2015}{Certified Tutor by the CRLA's International Tutor Training Program}
\cvitem{2015}{Trained in the Responsible Conduct of Research by the National Center for Professional \& Research Ethics}
\cvitem{2014--Present}{Peer Tutor for UM's Camner Academic Resource Center}


\end{document}
